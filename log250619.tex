\documentclass{amsart}

%%%%%%%%%%%%%%%

\usepackage[utf8]{inputenc}
% \usepackage[spanish]{babel}
% \usepackage[top=1in, bottom=1in, left=1.2in, right=1.2in]{geometry}
\usepackage{amssymb}
\usepackage{amsmath}
\usepackage{amsfonts}
\usepackage{amsthm}
\usepackage{wasysym}
\usepackage{enumitem}
\usepackage{graphicx}
\usepackage{listings}
\usepackage{xcolor}
\usepackage{tikz}

% sets
\newcommand{\NN}{\mathbf{N}}
\newcommand{\ZZ}{\mathbf{Z}}
\newcommand{\QQ}{\mathbf{Q}}
\newcommand{\RR}{\mathbf{R}}
\newcommand{\Zpos}{\ZZ^{+}}
\newcommand{\Rpos}{\RR^{+}}

% brackets
\newcommand{\la}{\langle}
\newcommand{\ra}{\rangle}

% formal statements
\newtheorem{prop}{Proposition}

\theoremstyle{plain}
\newtheorem{clm}{Claim}

\theoremstyle{definition}
\newtheorem{defn}{Definition}

\newtheorem{exl}{Example}

\theoremstyle{remark}
\newtheorem{rmk}{Remark}

% vulgar display of code

\lstdefinestyle{astyle}{
	commentstyle=\color{blue},
	keywordstyle=\color{purple},
	numberstyle=\tiny\color{gray},
	stringstyle=\color{green},
	basicstyle=\ttfamily\footnotesize,
	tabsize=2
}

\lstset{style=astyle}


\title{Work log of June 19}
\author{Daniel R. Barrero R.}
\date{\today}

\begin{document}

\maketitle

\section{}

The comonad defnition as a typeclass is

\lstinputlisting[language=Haskell, firstline=48, lastline=50]{DocCode.hs}

And in Milewski's tic-tac-toe, \emph{the evaluator} will be a comonad insntance,
apparently. This evaluator defines a tree-to-vector behavior:

\lstinputlisting[language=Haskell, firstline=59, lastline=59]{DocCode.hs}

The following typings follow:

\bigskip

The type of \texttt{runW} is

\lstinputlisting[language=Haskell, firstline=67, lastline=67]{DocCode.hs}

And that of \texttt{W} is

\lstinputlisting[language=Haskell, firstline=71, lastline=71]{DocCode.hs}

Also, \texttt{ghci} gives the following answer when asked for the kind of
\texttt{W}:

\lstinputlisting[language=Haskell, firstline=77, lastline=78]{DocCode.hs}

Since a \texttt{Comonad} is also a \texttt{Functor}, we must first make a
functor out of \texttt{W f}. For this, we first make \texttt{Vec n} into a
functor via

\lstinputlisting[language=Haskell, firstline=80, lastline=82]{DocCode.hs}

And then \texttt{W f} with

\lstinputlisting[language=Haskell, firstline=84, lastline=88]{DocCode.hs}

Now, to obtain a \texttt{Comonad} instance, we must define the \texttt{extract}
and \texttt{duplicate} maps. The first one is

\lstinputlisting[language=Haskell, firstline=90, lastline=91]{DocCode.hs}

Given the functor he has defined, as well as the \texttt{extract} function
above, one would expect the type of his \texttt{duplicate} to be

\lstinputlisting[language=Haskell, firstline=93, lastline=93]{DocCode.hs}

\end{document}
