\documentclass{amsart}

%%%%%%%%%%%%%%%

\usepackage[utf8]{inputenc}
% \usepackage[spanish]{babel}
% \usepackage[top=1in, bottom=1in, left=1.2in, right=1.2in]{geometry}
\usepackage{amssymb}
\usepackage{amsmath}
\usepackage{amsfonts}
\usepackage{amsthm}
\usepackage{wasysym}
\usepackage{enumitem}
\usepackage{graphicx}
\usepackage{listings}
\usepackage{xcolor}
\usepackage{tikz}

% sets
\newcommand{\NN}{\mathbf{N}}
\newcommand{\ZZ}{\mathbf{Z}}
\newcommand{\QQ}{\mathbf{Q}}
\newcommand{\RR}{\mathbf{R}}
\newcommand{\Zpos}{\ZZ^{+}}
\newcommand{\Rpos}{\RR^{+}}

% brackets
\newcommand{\la}{\langle}
\newcommand{\ra}{\rangle}

% formal statements
\newtheorem{prop}{Proposition}

\theoremstyle{plain}
\newtheorem{clm}{Claim}

\theoremstyle{definition}
\newtheorem{defn}{Definition}

\newtheorem{exl}{Example}

\theoremstyle{remark}
\newtheorem{rmk}{Remark}

% vulgar display of code

\lstdefinestyle{astyle}{
	commentstyle=\color{blue},
	keywordstyle=\color{purple},
	numberstyle=\tiny\color{gray},
	stringstyle=\color{green},
	basicstyle=\ttfamily\footnotesize,
	tabsize=2
}

\lstset{style=astyle}


\title{Monad given by an operad}
\author{Daniel R. Barrero R.}
\date{\today}

\begin{document}

\maketitle

\section{}

Let $f$ be an operad\footnote{Over the category $\mathcal{C}$ used by May: differential
$\ZZ-$graded $k-$modules.}, and for each object $X$ let $X^j$ denote its $j-$th tensor
power. Then, let $\mathtt{f} : \mathcal{C} \to \mathcal{C}$ be given by

$$
\mathtt{f}\ X = \bigoplus_{j \geq 0} f(j) \otimes_{k[\Sigma_j]} X^j
$$

It can be given a monad structure via the maps $u : X \to \mathtt{f}\ X$ (unit/\texttt{return})
and $\mu : \mathtt{f} \ (\mathtt{f} \ X) \to \mathtt{f} \ X$ (multiplication/\texttt{join}).

\bigskip

\noindent The unit map is given by $u(x) = \fone \otimes x$, where $\fone \in f(1)$ denotes the
composition identity element. The multiplication map $\mu$, on the other hand, is given by the
composition

\begin{displaymath}
	\begin{tikzcd}
		f(n) \otimes_{k[G]}
		\left( 
			\left( 
				f(j_1) \otimes_{k[G]} X^{j_1}
			\right) \otimes \cdots \otimes
			\left( 
				f(j_n) \otimes_{k[G]} X^{j_n}
			\right)
		\right) \arrow[d, "\text{shuffle}"] \\
		\left( 
			f(n) \otimes f(j_1) \otimes \cdots \otimes f(j_n)
		\right) \otimes X^j \arrow[d, "\gamma \otimes \mathrm{id}"] \\
		f(j) \otimes_{k[G]} X^j.
	\end{tikzcd}
\end{displaymath}

We used $G$ and $j$ above to mean the symmetric group of appropriate rank and $\sum_{s=1}^{n} j_s$,
respectively.

\section{Tensor product over non-commutative rings.}
The definition of the monad associated to an operad involves tensor product over the group ring
$k[\Sigma_j]$, which is non-commutative.  I will write down what I think the definitions should
be\footnote{After confirming these, I will hopefully also add some \emph{motivation} for their
use.}.

\begin{defn}
	If the ring $k$ is commutative, then there is no distinction between left- and right-
	modules. Therefore, the tensor product $A \otimes_k B$ of two $k-$modules is defined as
	$$
	k^{\oplus A \times B}/W,
	$$
	where $W$ is the sub-module generated by the elements
	\begin{eqnarray*}
		(a+a', b) - \left( (a, b) + (a', b) \right) \\
		(a, b+b') - \left( (a, b) + (a, b') \right) \\
		x(a, b) - (xa, b) \\
		x(a, b) - (a, xb).
	\end{eqnarray*}
	Then, we let $a \otimes b$ denote the equivalence class of $(a, b)$ in this quotient.
\end{defn}

However, if we let $R$ denote an arbitrary ring\footnote{Rings for us are fixed to have $1$, for
the more general case we use the term \emph{algebra}.} and we let $T$ denote some set, then the
``direct sum'' $R^{\oplus T}$ can only be guaranteed to make sense as an abelian group, since the
left- and right- module structures need not coincide.

\begin{defn}
	Let $R$ be a non-commutative ring, $A$ a right $R-$module, and $B$ a left $R-$module.
	Define their \emph{tensor product over $R$} as
	$$
	A \otimes_R B := \ZZ^{\oplus A \times B}/W,
	$$
	where $W$ is the subgroup generated by the elements
	\begin{eqnarray*}
		(a+a', b) - \left( (a, b) + (a', b) \right) \\
		(a, b+b') - \left( (a, b) + (a, b') \right) \\
		(ar, b) - (a, rb) \\
	\end{eqnarray*}
\end{defn}

In the case where the non-commutative ring $R$ contains a ``nice'' commutative ring $k$ as a
subring ---as in differential operators and group rings---, the tensor product is defined as the
quotient of the free $k-$module modulo the analogous submodule. Explicitly, the relations for
$A \otimes_{k[G]} B$ are those of $k-$bilinearity and

$$
(ag, b) - (a, gb),
$$

where $A$ is a $k-$module with a right $G-$action and $B$ is a $k-$module with a left $G-$action.

\subsection{Questions}

\begin{itemize}
	\item What is $R \otimes_R R$ when $R$ is not commutative?
	\item What is the defining mapping property of the abelian group $A \otimes_R B$ when $R$
		is not commutative?
\end{itemize}

\section{Differential-graded (dg) modules}

A dg $k-$module $(X_\cdot, d_\cdot)$ consists of the data $X_\cdot = (X_n : k-\mathbf{mod})_{n \in \ZZ}$ and 
differential maps $d_\cdot = (d_n : X_n \to X_{n-1})_{n \in \ZZ}$ that are $k-$linear and $d_{n-1} \circ d_n = 0$.
This is usually denoted $d^2 = 0$. When the context yields enough understanding, $(X_\cdot, d_\cdot)$ will be
abbreviated $X$, and other analogous abbreviations follow.

\begin{defn}
	A map $f : X \to Y$ of dg modules is the data of maps $f_n : X_n \to Y_n$ for each $n \in \ZZ$, such
	that $df = fd$, namely $d_n^Y \circ f_n = f_{n-1} \circ d_n^X$, for each $n \in \ZZ$.
\end{defn}

\end{document}
