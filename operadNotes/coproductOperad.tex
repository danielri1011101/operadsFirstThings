\documentclass{amsart}

%%%%%%%%%%%%%%%

\usepackage[utf8]{inputenc}
% \usepackage[spanish]{babel}
% \usepackage[top=1in, bottom=1in, left=1.2in, right=1.2in]{geometry}
\usepackage{amssymb}
\usepackage{amsmath}
\usepackage{amsfonts}
\usepackage{amsthm}
\usepackage{wasysym}
\usepackage{enumitem}
\usepackage{graphicx}
\usepackage{listings}
\usepackage{xcolor}
\usepackage{tikz}

% sets
\newcommand{\NN}{\mathbf{N}}
\newcommand{\ZZ}{\mathbf{Z}}
\newcommand{\QQ}{\mathbf{Q}}
\newcommand{\RR}{\mathbf{R}}
\newcommand{\Zpos}{\ZZ^{+}}
\newcommand{\Rpos}{\RR^{+}}

% brackets
\newcommand{\la}{\langle}
\newcommand{\ra}{\rangle}

% formal statements
\newtheorem{prop}{Proposition}

\theoremstyle{plain}
\newtheorem{clm}{Claim}

\theoremstyle{definition}
\newtheorem{defn}{Definition}

\newtheorem{exl}{Example}

\theoremstyle{remark}
\newtheorem{rmk}{Remark}

% vulgar display of code

\lstdefinestyle{astyle}{
	commentstyle=\color{blue},
	keywordstyle=\color{purple},
	numberstyle=\tiny\color{gray},
	stringstyle=\color{green},
	basicstyle=\ttfamily\footnotesize,
	tabsize=2
}

\lstset{style=astyle}


\title{Coproduct of Operads}
\author{Daniel R. Barrero R.}
\date{\today}

\begin{document}

\maketitle

\section{Free operads}

Let $U : \mathrm{Op} \to \mathrm{Top}^{\NN}$ be the forgetful functor and let $F$ be its
left adjoint. Its existence is a result of Boardman and Vogt in \cite{bv-hiasots}.

\section{The construction}

Thanks to de $F,U$ adjoint pair there exists an epimorphism

\begin{equation}\label{fu-epi}
	\epsilon : FU \ O \to O
\end{equation}


for each operad $O$.

Given two operads $O$ and $O'$, and drawing inspiration\footnote{We haven't established nor cited
any relevant categorical properties for operads.} from the fact that left adjoints
preserve colimits \cite{riehl-ctic}, we may define

\begin{equation}\label{fu-coprod}
	(FU \ O) \ + \ (FU \ O') \ := \ F \ \left( (U \ O) \ + \ (U \ O') \right)
\end{equation}

\section{Comments}

\subsection{} The result by B \& V is that the forgetful functor is \emph{monadic.}

\subsection{} The full statement that motivates \eqref{fu-coprod} is that
\emph{Right adjoints preserve limits and left adjoints preserve colimits.} It can be found
in \cite{riehl-ctic}.

\bibliographystyle{acm}
\bibliography{refs}

\end{document}
