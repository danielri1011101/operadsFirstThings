\documentclass{amsart}

%%%%%%%%%%%%%%%

\usepackage[utf8]{inputenc}
% \usepackage[spanish]{babel}
% \usepackage[top=1in, bottom=1in, left=1.2in, right=1.2in]{geometry}
\usepackage{amssymb}
\usepackage{amsmath}
\usepackage{amsfonts}
\usepackage{amsthm}
\usepackage{wasysym}
\usepackage{enumitem}
\usepackage{graphicx}
\usepackage{listings}
\usepackage{xcolor}
\usepackage{tikz}

% addresses

\newcommand{\picPrefix}
{/home/profesor/danielr/doctorado/B-operads/pictures}

% sets
\newcommand{\NN}{\mathbf{N}}
\newcommand{\ZZ}{\mathbf{Z}}
\newcommand{\QQ}{\mathbf{Q}}
\newcommand{\RR}{\mathbf{R}}
\newcommand{\Zpos}{\ZZ^{+}}
\newcommand{\Rpos}{\RR^{+}}

% brackets
\newcommand{\la}{\langle}
\newcommand{\ra}{\rangle}

% formal statements
\newtheorem{prop}{Proposition}

\theoremstyle{plain}
\newtheorem{clm}{Claim}

\theoremstyle{definition}
\newtheorem{defn}{Definition}

\newtheorem{exl}{Example}

\theoremstyle{remark}
\newtheorem{rmk}{Remark}

% vulgar display of code

\lstdefinestyle{astyle}{
	commentstyle=\color{blue},
	keywordstyle=\color{purple},
	numberstyle=\tiny\color{gray},
	stringstyle=\color{green},
	basicstyle=\ttfamily\footnotesize,
	tabsize=2
}

\lstset{style=astyle}


\title{Coproduct of Operads}
\author{Daniel R. Barrero R.}
\date{\today}

\begin{document}

% -	C.f. work log of June 28	- %

\maketitle

\section{Trees}

\begin{defn}\label{def-tree}
	An $n-$\emph{tree} is the data $(V, E, s, t)$ where
	\begin{itemize}
		\item $V$ is a finite set of \emph{vertices},
		\item $E$ is a finite set of \emph{edges},
		\item $s : E \to V \cup [1..n]$ is the
			\emph{source map}\footnote{Here we're using the
			comprehension $[a..b] :=
			\{t \in \ZZ \ | \ a \leq t \leq b\}$}, and
		\item $t : E \to V \cup \{0\}$ is the \emph{target map}.
	\end{itemize}
	This data is restricted to satisfy the following conditions:
	\begin{enumerate}
		\item It defines a graph-theoretic tree with vertices
			$V \cup [0..n]$ and edges $E$.
		\item There is exactly one $e \in E$ such that $t(e) = 0$.
	\end{enumerate}
	We use $u \to^{e} v$ to denote\footnote{Since the vertices and
	edges form a tree,  specifying the edge is redundant. We will keep
	this definition for the time being in order to fluidly converse
	with the authors.} that $e \in E$ has source $u$ and target $v$.
\end{defn}

\section{Free operads}

Let $U : \mathrm{Op} \to \mathrm{Top}^{\NN}$ be the forgetful functor and
let $F$ be its left adjoint. Its existence is a result of Boardman and Vogt
in \cite{bv-hiasots}.

\begin{defn}
	Let $C$ be a collection. A \emph{$C-$labelled planar $n-$tree} is a
	planar $n-$tree such that each vertex with $k$ children is labelled
	by an element of $C_k$.
\end{defn}

	

\section{The construction}

Given two operads $O$ and $O'$, the $F-U$ adjoint pair gives epimorphisms 

\begin{eqnarray}\label{fu-epis}
	\epsilon : FU \ O \to O \\
	\epsilon' : FU \ O' \to O'.
\end{eqnarray}

If the category $\mathrm{Op}$ had coproducts, we would have an operad
epimorphism\footnote{It being an epimorphism is not immediate, but the
property is necessary for the validity of the construction.}

\begin{equation}\label{cpd-epis}
	\epsilon + \epsilon' : (FU \ O) + (FU \ O') \to O + O'.
\end{equation}

Drawing inspiration\footnote{We haven't established nor cited any relevant
categorical properties for operads.} from the fact that left adjoints
preserve colimits \cite{riehl-ctic}, we may define


\begin{equation}\label{fu-coprod}
	(FU \ O) \ + \ (FU \ O') \ := \ F \ \left( (U \ O) \ + \ (U \ O')
	\right).
\end{equation}

and therefore compute a quotient of $F \ \left( (U \ O) + (U \ O')
\right)$ that is a coproduct of $O$ and $O'$.

\section{Comments}

\subsection{} The result by B \& V is that the forgetful functor is \emph{monadic.}

\subsection{} The full statement that motivates \eqref{fu-coprod} is that
\emph{Right adjoints preserve limits and left adjoints preserve colimits.}
It can be found in \cite{riehl-ctic}.

\subsection{} The definition \eqref{fu-coprod} suggests that
\emph{``at the $FU-$level coproducts are trivially definied.''}

\subsection{} Justify that $\epsilon + \epsilon'$ in \eqref{cpd-epis} is
an epimorphism.

\subsection{} The number of edges of a tree given by definition
\ref{def-tree} is probably

$$
(\# \ V) + n,
$$

where $n$ is the number of leaves.

\bibliographystyle{acm}
\bibliography{refs}

\end{document}
