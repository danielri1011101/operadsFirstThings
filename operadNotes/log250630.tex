\documentclass{amsart}

%%%%%%%%%%%%%%%

\usepackage[utf8]{inputenc}
% \usepackage[spanish]{babel}
% \usepackage[top=1in, bottom=1in, left=1.2in, right=1.2in]{geometry}
\usepackage{amssymb}
\usepackage{amsmath}
\usepackage{amsfonts}
\usepackage{amsthm}
\usepackage{wasysym}
\usepackage{enumitem}
\usepackage{graphicx}
\usepackage{listings}
\usepackage{xcolor}
\usepackage{tikz}

% addresses

\newcommand{\picPrefix}
{/home/profesor/danielr/doctorado/B-operads/pictures}

% sets
\newcommand{\NN}{\mathbf{N}}
\newcommand{\ZZ}{\mathbf{Z}}
\newcommand{\QQ}{\mathbf{Q}}
\newcommand{\RR}{\mathbf{R}}
\newcommand{\Zpos}{\ZZ^{+}}
\newcommand{\Rpos}{\RR^{+}}

% brackets
\newcommand{\la}{\langle}
\newcommand{\ra}{\rangle}

% formal statements
\newtheorem{prop}{Proposition}

\theoremstyle{plain}
\newtheorem{clm}{Claim}

\theoremstyle{definition}
\newtheorem{defn}{Definition}

\newtheorem{exl}{Example}

\theoremstyle{remark}
\newtheorem{rmk}{Remark}

% vulgar display of code

\lstdefinestyle{astyle}{
	commentstyle=\color{blue},
	keywordstyle=\color{purple},
	numberstyle=\tiny\color{gray},
	stringstyle=\color{green},
	basicstyle=\ttfamily\footnotesize,
	tabsize=2
}

\lstset{style=astyle}


\title{Work log of June 30}
\author{Daniel R. Barrero R.}
\date{\today}

\begin{document}

\maketitle

\section{Comments}

\subsection{} Let's say we have a category $\mathcal{C}$ on which we
regularly do ``operaton-like manipulations''. Then we may expect the
objects of $\mathcal{C}$ ---or some of them--- to be algebras of some
operad. Is this the \emph{raison d'etrè}?

\subsection{} Thm 23 of \cite{baezOtter} describes the free operad $F \ C$
of a collection.

\subsection{} Is he saying that the left adjoint to the forgetful functor
is the free operad functor?

\subsection{} Seems like for Baez' trees, the target map isn't injective
in the cases one cares about.

\subsection{} Drawing these trees vertically is what makes the distinction
between source and target meaningful.

% -	-	-	-	-	-	-	-	- %
% % 	*Wisdom consists of knowing when to avoid perfection.*  % %
% -	-	-	-	-	-	-	-	- %

\bibliographystyle{acm}
\bibliography{refs}

\end{document}
