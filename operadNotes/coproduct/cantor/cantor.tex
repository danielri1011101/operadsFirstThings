The main reference is the exercises segment from \S 6.6 of the book \cite{rosen7thEd}.

The idea of Cantor digits consists of representing a permutation of $[1..n]$ with
a list of non-negative integers of length $n-1$, with repetitions allowed.

\bigskip
**\emph{Display explicit examples for small values of $n$, that suggest the
computational efficiency}**

\bigskip
Specifically, we have

\begin{defn}
	Let $n > 1$ and let $k \in [0..n!-1]$. The \emph{Cantor expansion} of
	$k$ is the list $[a_1, \cdots, a_n]$ where the numbers $a_i$ are the unique
	integers such that $0 \leq a_i \leq i$ and
	\[
		k = \sum_{i= 1}^{n-1} a_ii!
	\]
	The integers $a_1,\cdots,a_{n-1}$ are called the \emph{Cantor digits} of
	$k$.
\end{defn}

Clearly, this definition is preceded by the following theorem.

\begin{thm}\label{ce} % Cantor expansion
	Let $n > 1$ and let $k \in [0..n!-1]$. Then there exist unique integers
	$a_1,\cdots,a_{n-1}$ such that $0 \leq a_i \leq i$ and
	\[
		k = \sum_{i= 1}^{n-1} a_ii!
	\]
\end{thm}
%
\begin{proof}
	Let $n= 2$. Then the expressions $0 = 0\cdot1!$ and $1 = 1\cdot1!$ are
	clearly true and unique.
	Now let $n > 2$ and suppose that for all $k \in [0..(n-1)!-1]$ there exist
	unique integers $a_1,\cdots,a_{n-2}$ such that
	\[
		\begin{cases}
			0 \leq a_i \leq i\\
			k = \sum_{i= 1}^{n-2} a_ii! 
		\end{cases}
	\]
	Let $K \in [(n-1)!..n!-1]$. Then, there exist unique non-negative integers
	$k$ and $Q$ such that
	\[
		\begin{cases}
			K = k + Q\\
			k \in [0..(n-1)!-1]\\
			Q \in [(n-1)!..(n-1)!(n-1)!]
		\end{cases}
	\]
	Explicitly, $Q = q(n-1)!$ given by the division algorithm. Also notice
	that $q < n$ because $K < n!$.
	Therefore
	\[
		K = \sum_{i= 1}^{n-1}b_ii!,
	\]
	where for $i < n-1$ we let $b_i= a_i$ given by the induction hypothesis
	applied to $k$ and we let $b_{n-1}= q$.

	To see the uniqueness of these Cantor digits, suppose
	\[
		K = \sum_{i= 1}^{n-1} c_ii!
	\]
	for coefficients $c_i$ satisfying the appropriate restrictions. Then
	\[
		K = c_{n-1}(n-1)! + \sum_{i=1}^{n-2} c_ii!
	\]
	Thanks to lemma \ref{mcs}, $\sum_{i= 1}^{n-2} c_ii! < (n-1)!$, and therefore
	$c_i = b_i$ for all $i$.
\end{proof}

This lemma is relevant to the proof of theorem \ref{ce}.
\begin{lma}\label{mcs} % Max Cantor sum
	Let $n > 1$. Then
	\[
		\sum_{i= 1}^{n-1} i\cdot i! = n!-1.
	\]
\end{lma}
