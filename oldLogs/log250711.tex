\documentclass{amsart}

%%%%%%%%%%%%%%%

\usepackage[utf8]{inputenc}
% \usepackage[spanish]{babel}
% \usepackage[top=1in, bottom=1in, left=1.2in, right=1.2in]{geometry}
\usepackage{amssymb}
\usepackage{amsmath}
\usepackage{amsfonts}
\usepackage{amsthm}
\usepackage{wasysym}
\usepackage{enumitem}
\usepackage{listings}
\usepackage{xcolor}
\usepackage{tikz}

% sets
\newcommand{\NN}{\mathbf{N}}
\newcommand{\ZZ}{\mathbf{Z}}
\newcommand{\QQ}{\mathbf{Q}}
\newcommand{\RR}{\mathbf{R}}
\newcommand{\Zpos}{\ZZ^{+}}
\newcommand{\Rpos}{\RR^{+}}

% brackets
\newcommand{\la}{\langle}
\newcommand{\ra}{\rangle}

% formal statements
\newtheorem{prop}{Proposition}

\theoremstyle{plain}
\newtheorem{clm}{Claim}

\theoremstyle{definition}
\newtheorem{defn}{Definition}

\newtheorem{exl}{Example}

\theoremstyle{remark}
\newtheorem{rmk}{Remark}

% display of code
\definecolor{oliveGreen}{rgb}{0.13,0.55,0.13}
\lstset{%
    language = Java,
    basicstyle = \normalsize\ttfamily,
    keywordstyle = \bf,
    commentstyle = \color{oliveGreen}\it,
    stringstyle = \color{blue},
    showstringspaces = false,
    columns = fullflexible%
}


\title{Work log of June 11 2025}
\author{Daniel R. Barrero R.}
% \date{\today}

\begin{document}

\maketitle

% No idea how to use these <++> place holders.
Existence types are defined (by Pierce) as  

$$
\exists X . T := \forall Y . (\forall X . T \to Y) \to Y.
$$

We can provide the following \emph{informal} validation of it in
terms of classical logic:

\begin{eqnarray*}
	\exists X . T \vdash \\
	\neg (\forall X . \neg T) \vdash \\
	\neg (\forall X . \neg T \lor Y) \lor Y \vdash \\
	(\forall X . T \to Y) \to Y \vdash \\
	\forall Y . (\forall X . T \to Y) \to Y
\end{eqnarray*}

\section{comments}

\begin{itemize}
	\item Realized that the singleton type family \texttt{SNat n} makes the
		\texttt{Nat} kind inhabitted.
	\item I propose attempting to use \textbf{where} syntax to improve the
		current readability given by \textbf{let} bindings.
	\item At first glance, the most appropriate continuation to be the last
		argument of \texttt{splitForest} is identity.
\end{itemize}

\end{document}
