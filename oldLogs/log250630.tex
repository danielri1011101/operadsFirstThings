\documentclass{amsart}

%%%%%%%%%%%%%%%

\usepackage[utf8]{inputenc}
% \usepackage[spanish]{babel}
% \usepackage[top=1in, bottom=1in, left=1.2in, right=1.2in]{geometry}
\usepackage{amssymb}
\usepackage{amsmath}
\usepackage{amsfonts}
\usepackage{amsthm}
\usepackage{wasysym}
\usepackage{enumitem}
\usepackage{graphicx}
\usepackage{listings}
\usepackage{xcolor}
\usepackage{tikz}

% sets
\newcommand{\NN}{\mathbf{N}}
\newcommand{\ZZ}{\mathbf{Z}}
\newcommand{\QQ}{\mathbf{Q}}
\newcommand{\RR}{\mathbf{R}}
\newcommand{\Zpos}{\ZZ^{+}}
\newcommand{\Rpos}{\RR^{+}}

% brackets
\newcommand{\la}{\langle}
\newcommand{\ra}{\rangle}

% formal statements
\newtheorem{prop}{Proposition}

\theoremstyle{plain}
\newtheorem{clm}{Claim}

\theoremstyle{definition}
\newtheorem{defn}{Definition}

\newtheorem{exl}{Example}

\theoremstyle{remark}
\newtheorem{rmk}{Remark}

% vulgar display of code

\lstdefinestyle{astyle}{
	commentstyle=\color{blue},
	keywordstyle=\color{purple},
	numberstyle=\tiny\color{gray},
	stringstyle=\color{green},
	basicstyle=\ttfamily\footnotesize,
	tabsize=2
}

\lstset{style=astyle}


\title{Work log of June 30}
\author{Daniel R. Barrero R.}
\date{\today}

\begin{document}

\maketitle

\section{Comments}

\subsection{} Let's say we have a category $\mathcal{C}$ on which we
regularly do ``operaton-like manipulations''. Then we may expect the
objects of $\mathcal{C}$ ---or some of them--- to be algebras of some
operad. Is this the \emph{raison d'etrè}?

\subsection{} Thm 23 of \cite{baezOtter} describes the free operad $F \ C$
of a collection.

\subsection{} Is he saying that the left adjoint to the forgetful functor
is the free operad functor?

\subsection{} Seems like for Baez' trees, the target map isn't injective
in the cases one cares about.

\subsection{} Drawing these trees vertically is what makes the distinction
between source and target meaningful.

% -	-	-	-	-	-	-	-	- %
% % 	*Wisdom consists of knowing when to avoid perfection.*  % %
% -	-	-	-	-	-	-	-	- %

\subsection{} Is it correct to say that \cite{baezOtter} discusses
the formal properties of a mathematical model?

\subsection{} If the edges of $n-$trees are given lengths, then they
naturally become planar. Nonetheless, the \emph{child of} property should
not ``propagate'' in order for this to hold.

\subsection*{} Indeed, the \emph{children} of a vertex are the edges of
which it is the target.

% % - I'm cultivating the opinion that definitions are meant for
% % - _organizing information_, and therefore one need not worry too much
% % - about changing them or improving them, as understanding increases.
% % - Especially in ``raw works'' like Baez', the definitions haven't been
% % - _peer reviewed_, and arguably they must be before they become _widely
% % - accepted_, or harder to dispute.

\subsection{} *IMPORTANT CRITICISM*: it is a problem that definition 7 is
the immediate follow-up of definition 6, since ``isomorphism of $n-$trees
\emph{with lengths}'' is undefined. Therefore, the reader may assume it is
an isomorphism of the underlying $n-$trees.

\subsection{} This would be fixed if they let a tree with lengths be planar
in the obvious way.

\bibliographystyle{acm}
\bibliography{refs}

\end{document}
