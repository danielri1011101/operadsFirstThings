\documentclass{amsart}

%%%%%%%%%%%%%%%

\usepackage[utf8]{inputenc}
% \usepackage[spanish]{babel}
% \usepackage[top=1in, bottom=1in, left=1.2in, right=1.2in]{geometry}
\usepackage{amssymb}
\usepackage{amsmath}
\usepackage{amsfonts}
\usepackage{amsthm}
\usepackage{wasysym}
\usepackage{enumitem}
\usepackage{graphicx}
\usepackage{listings}
\usepackage{xcolor}
\usepackage{tikz}

% addresses

\newcommand{\picPrefix}
{/home/profesor/danielr/doctorado/B-operads/pictures}

% sets
\newcommand{\NN}{\mathbf{N}}
\newcommand{\ZZ}{\mathbf{Z}}
\newcommand{\QQ}{\mathbf{Q}}
\newcommand{\RR}{\mathbf{R}}
\newcommand{\Zpos}{\ZZ^{+}}
\newcommand{\Rpos}{\RR^{+}}

% brackets
\newcommand{\la}{\langle}
\newcommand{\ra}{\rangle}

% formal statements
\newtheorem{prop}{Proposition}

\theoremstyle{plain}
\newtheorem{clm}{Claim}

\theoremstyle{definition}
\newtheorem{defn}{Definition}

\newtheorem{exl}{Example}

\theoremstyle{remark}
\newtheorem{rmk}{Remark}

% vulgar display of code

\lstdefinestyle{astyle}{
	commentstyle=\color{blue},
	keywordstyle=\color{purple},
	numberstyle=\tiny\color{gray},
	stringstyle=\color{green},
	basicstyle=\ttfamily\footnotesize,
	tabsize=2
}

\lstset{style=astyle}


\title{Work log of June 28}
\author{Daniel R. Barrero R.}
\date{\today}

\begin{document}

\maketitle

\section{}

The free operads of Baez are obtained via $C-$trees, where $C$ is a
\emph{collection}.

\begin{defn}
	A \emph{$C-n-$tree} is a planar $n-$tree such that each vertex
	with $k$ children is labeled by an element of $C_k$.
\end{defn}

\section{}

The identification of phylogentetic trees with the operations of
$\mathrm{Com} + \Rsemi$ is thanks to the following theorem in
\cite{baezOtter}:

\begin{thm}\label{bo40}
	If $O$ and $O'$ are operads such that $O'$ has only unary
	operations, then there is a bijection between $(O + O')_n$ and
	equivalence classes of $(O,O'_1)-n-$trees such that no unary vertex
	is labeled with $1_O$ and no internal edge is labeled with
	$1_{O'}$.
\end{thm}

\subsection{} The equivalence relation considered in theorem \ref{bo40} is
the one given by the \emph{natural} way to permute labeled rooted trees,
namely respecting ``genealogy'' or, more formally, its \emph{partial order
structure}.

% -							       - %
% --------------- Closing section of comments ------------------ %
% -							       - %

\section{Comments}

\subsection{} The definition of \emph{operad} in \cite{baezOtter}
coincides with that of May in \emph{Geometry of iterated loop spaces}.
Namely, it is a symmetric operad in the symmetric monoidal category of
topological spaces. With the exception that more than one 0-ary operations
are allowed.

\subsection{} The forgetful functor

$$
U : \mathrm{Op} \to \mathrm{Top}^\NN
$$

is natural because operad maps are degree-preserving.

\subsection{} Operad maps are defined by preserving the commutative
diagrams of operads, and being continuous\footnote{Read May in case there
are topological or otherwise considerations I'm ignoring.}.

\bibliographystyle{acm}
\bibliography{\refsHomePrefix/refs.bib}



\end{document}
